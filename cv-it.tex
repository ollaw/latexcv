% !TEX encoding = UTF-8
% !TEX program = pdflatex
% !TEX spellcheck = it_IT

\documentclass[italian,a4paper]{europasscv}
\usepackage[italian]{babel}
\usepackage{comment}
\usepackage{lastpage}
\ecvname{Olla Gabriele}
\ecvaddress{Via Nizza 356bis, Torino (TO)}
\ecvmobile{+39 348 6011 931}

\ecvemail{ollagabriele@pm.me}
\ecvgitlabpage{ollaww}
\ecvgithubpage{ollaw}

\ecvgender{Uomo}
\ecvdateofbirth{17/02/1995}
\ecvnationality{Italiana}




\begin{document}
\begin{europasscv}
			
	\ecvpersonalinfo
		
	\vspace{.75cm}
			
	\ecvsection{Esperienze lavorative}
			  
	\ecvtitle{Settembre 2022 -- Ora }{Site Relialibity Engineer}
	\ecvitem{}{\textbf{Prima Assicurazioni}, Milano [\textit{Full remote}]}
	\ecvitem[2.5mm]{}{
		\begin{ecvitemize}
			\item Split, migrazione e gestione di molteplici account e infrastrutture su AWS (Pulumi);
			\item Amministrazione di cluster Kubernetes (EKS, Istio);
			\item Design e implementazione di toolchain CI/CD (Github Actions, Helm);
			\item Monitoring dell'infrastruttura (Datadog);
			\item Gestione DNS e CDN (Cloudflare).
		\end{ecvitemize}
	}
	\vspace{0.5cm}
			    
	\ecvtitle{Maggio 2022 -- Settembre 2022 }{DevOps Engineer}
	\ecvitem{}{\textbf{Prima Assicurazioni}, Milano [\textit{Full remote}]}
	\ecvitem[2.5mm]{}{
		\begin{ecvitemize}
			\item Amministrazione di cluster Kubernetes (EKS) ed ECS;
			\item Design ed implementazione di infrastrutture (AWS, Pulumi, CloudFormation);
			\item Design ed implementazione di processi di rilascio (Drone, Helm).
			\item Monitoring dell'infrastruttura (ELK, Datadog).
		\end{ecvitemize}
	}
	\vspace{0.5cm}
			  
	\ecvtitle{Settembre 2021 -- Maggio 2022}{Cloud Engineer}
	\ecvitem{}{\textbf{DXC Technology}, Cernusco sul Naviglio (MI) [\textit{Lavoro ibrido}]}
	\ecvitem[2.5mm]{}{
		\begin{ecvitemize}
			\item Installazione e amminsitrazione di cluster Kubernetes (MKE, RKE\dots);
			\item Creazione e automazione di infrastrutture su OpenStack (Terraform, Ansible);
			\item Implementazione di processi di CI/CD(Gitlab, Docker Swarm, Kubernetes);
			\item Troubleshooting su Linux VMs.
		\end{ecvitemize}
	}
	\vspace{0.5cm}
			
	\ecvtitle{Marzo 2020 -- Agosto 2021}{Cloud Engineer}
	\ecvitem{}{\textbf{Storm Reply}, Torino [\textit{Lavoro ibrido}]}
	\ecvitem[2.5mm]{}{
		\begin{ecvitemize}
			\item Design e implementazione di infrasrutture su AWS;
			\item Creazione di toolchains DevOps (Jenkins, EKS\dots);
			\item Deploy di piattaforme di logging e monitoring (Grafana, Prometheus, ELK\dots).
		\end{ecvitemize}
	}
	\vspace{0.5cm}
		
	\pagebreak
			           
	\ecvsection{Educazione e formazione}
			  
	\ecvtitlelevel{2017 -- 2020}{Laurea magistrale in scienze e tecnologie informatiche}{ISCED~7}
	\ecvitem{}{\textbf{Università degli studi di Torino}}
	\ecvitem[.2cm]{Votazione}{ 110/110 con Lode e menzione }
	\ecvitem{Tesi}{Identity and Access governance : gestione delle identità digitali all'interno di un'azienda.}
			  
			  
	\ecvtitlelevel{2014 -- 2017}{Laurea triennale in scienze e tecnologie informatiche}{ISCED~6}
	\ecvitem{}{\textbf{Università degli studi di Torino}}
	\ecvitem[.2cm]{Votazione}{ 110/110 con Lode}
	\ecvitem{Tesi}{Vulnerability assessment di applicazioni Android: uno strumento per l'automazione.}
			
	\vspace{1cm}
		 
	\ecvsection{Competenze tecniche}
	\ecvblueitem{Cloud}{Ampia familiarità ed esperienza con AWS oltre che una conoscenza base di Azure,  GCP ed OpenStack. Ottima conoscenza di diversi strumenti per  IAC tra cui \textit{Pulumi}, \textit{Terraform} e CloudFormation (e \textit{SAM}) }
		
	\ecvblueitem{Container e orchestrazione}{Ottima competenze in ambito container ed eccellente conoscenza di Kubernetes, sia in modalità \textit{managed} (EKS,AKS) che \textit{self-managed} (MKE,RKE), oltre che di alcuni dei principali strumenti affini come \textit{Helm}, \textit{Istio} e \textit{Kustomize}.}
		
	\ecvblueitem{Versioning e CI/CD}{Buona conoscenza di \textit{Git} ed ampia esperienza nell'utilizzo di \textit{GitHub} e \textit{GitLab}.Ottime competenze nell'analisi e nell'implementazione di processi di CI/CD tramite strumenti di automazione tra cui \textit{Gitlab CI}, \textit{GitHub Actions}, \textit{Harness Drone}, \textit{Jenkins} (sfruttando l'approccio scalabile \textit{Jenkins on Kubernetes}), e \textit{Bamboo}. Conoscenza ed esperienza del paradigma \textit{GitOps} e della sua applicazione con strumenti quali \textit{Flux} e \textit{GitLab Agent}.}
		
	\ecvblueitem{Linguaggi}{Ottima conoscenza di \textit{Java}, \textit{Kotlin} e \textit{Python} oltre che una conoscenza universitaria di diversi linguaggi tra cui C (+ OpenMP/MPI) e JS (+ \textit{React}/\textit{VUE}). Buone competenze in ambito scripting tramite \textit{Bash} ed esperienza universitaria in sviluppo mobile (Android SDK).}
		
	\ecvblueitem{Altro}{Buona conoscenza dei RDBMS (e del linguaggio SQL), dei principali sistemi di observability (\textit{Datadog}, \textit{Prometheus}, \textit{Grafana} e dello stack \textit{ELK}) oltre che dei sistemi Linux.}
		
	\vspace{1cm}
		
	\ecvsection{Lingua}
	\ecvmothertongue{Italiano}
	\ecvblueitem{Altre lingue}{
		Inglese (C1 - Fluentify Language Assessment 12/2022)
	}
			
			
	\pagebreak
		
	\ecvsection{Certificazioni}
			    
	\ecvblueitem{Dicembre 2021}{Certified Kubernetes Application Developer (CKAD)}
	\ecvitem[.2cm]{Ente certificatore}{Cloud Native Computing Foundation (CNCF)}
	\ecvitem{Identificativo}{ LF-owlpgjxmun (https://training.linuxfoundation.org/certification/verify/)}\hfill
	\smash{\includegraphics[width=1.5cm]{img/ckad-certified-kubernetes-application-developer.png}}
			        
	\ecvblueitem{Novembre 2021}{Certified Kubernetes Administrator (CKA)}
	\ecvitem[.2cm]{Ente certificatore}{Cloud Native Computing Foundation (CNCF)}
	\ecvitem{Identificativo}{ LF-4w8nbsr5xc (https://training.linuxfoundation.org/certification/verify/)}\hfill
	\smash{\includegraphics[width=1.5cm]{img/cka-certified-kubernetes-administrator.png}}
			        
	\ecvblueitem{Febbraio 2021}{AWS Certified Solution Architect - Associate (SAA-CO2)}
	\ecvitem[.2cm]{Ente certificatore}{Amazon Web Services (AWS)}
	\ecvitem{Identificativo}{2898P6CBJ111QMWC (http://aws.amazon.com/verification)}\hfill
	\smash{\includegraphics[width=1.5cm]{img/aws-certified-solutions-architect-associate.png}}
			  
	\vspace{1cm}
		   
	\ecvsection{Ulteriori informazioni}
			    
	\ecvblueitem{Patente}{Patente B}
			        
	\ecvblueitem{Hobbies e interessi}{
		Nel tempo libero amo fare sport, in particolare sciare, trekking o giocare a calcio.\newline Inoltre mi piace molto leggere, giocare a scacchi, curare l'acquario e fare giardinaggio.
	}
			        
	\ecvblueitem{Trattamento dei dati}{Autorizzo il trattamento dei miei dati personali ai sensi del Decreto Legislativo 30 giugno 2003, n. 196 "Codice in materia di protezione dei dati personali.}
			
			  
\end{europasscv}

\end{document}
