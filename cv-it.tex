% !TEX encoding = UTF-8
% !TEX program = pdflatex
% !TEX spellcheck = en_GB

\documentclass[italian,a4paper]{europasscv}
\usepackage[italian]{babel}
\usepackage{comment}
\usepackage{lastpage}
\ecvname{Olla Gabriele}
\ecvaddress{Via Nizza 356bis, Torino (TO)}
\ecvmobile{+39 348 6011 931}

\ecvemail{ollagabriele@pm.me}
\ecvgitlabpage{https://gitlab.com/ollaww}
\ecvgithubpage{https://github.com/ollaw}
\ecvlinkedinpage{https://www.linkedin.com/in/ollagabriele}

\ecvgender{Uomo}
\ecvdateofbirth{17/02/1995}
\ecvnationality{Italiana}


\begin{document}
  \begin{europasscv}

  \ecvpersonalinfo

  \ecvsection{Esperienze lavorative}
  
    \ecvtitle{Settembre 2022 -- Ora }{Site Relialibity Engineer}
    \ecvitem{}{\textbf{Prima Assicurazioni}, Milano [\textit{Full remote}]}
    \ecvtitle{Maggio 2022 -- Settembre 2022 }{DevOps Engineer}
    \ecvitem{}{\textbf{Prima Assicurazioni}, Milano [\textit{Full remote}]}
  
    \ecvtitle{Settembre 2021 -- Maggio 2022}{Cloud Engineer}
    \ecvitem{}{\textbf{DXC Technology}, Cernusco sul Naviglio (MI) [\textit{Lavoro ibrido}]}
    \ecvitem[2.5mm]{}{
    \begin{ecvitemize}
        \item Installazione e amminsitrazione di cluster Kubernetes (MKE, RKE\dots);
        \item Creazione di infrastrutture su OpenStack;
        \item Implementazione di processi di CI/CD(Docker Swarm, Kubernetes);
        \item Troubleshooting su Linux VMs.
      \end{ecvitemize}
    }

    \ecvtitle{Marzo 2020 -- Agosto 2021}{AWS Cloud Engineer}
    \ecvitem{}{\textbf{Storm Reply}, Torino [\textit{In sede}]}
    \ecvitem[2.5mm]{}{
    \begin{ecvitemize}
        \item Design e implementazione di infrasrutture su AWS;
        \item Creazione di toolchains DevOps per ambienti di produzione (Jenkins, EKS\dots);
        \item Piattaform di logging e monitoring (Grafana, Prometheus, ELK\dots).
    \end{ecvitemize}
    }
            
    \ecvsection{Certificazioni}
    
        \ecvblueitem{Dicembre 2021}{Certified Kubernetes Application Developer (CKAD)}
        \ecvitem[.2cm]{Ente certificatore}{Cloud Native Computing Foundation (CNCF)}
        \ecvitem{Identificativo}{ LF-owlpgjxmun (https://training.linuxfoundation.org/certification/verify/)}\hfill
        \smash{\includegraphics[width=1.5cm]{img/ckad-certified-kubernetes-application-developer.png}}
        
        \ecvblueitem{Novembre 2021}{Certified Kubernetes Administrator (CKA)}
        \ecvitem[.2cm]{Ente certificatore}{Cloud Native Computing Foundation (CNCF)}
        \ecvitem{Identificativo}{ LF-4w8nbsr5xc (https://training.linuxfoundation.org/certification/verify/)}\hfill
        \smash{\includegraphics[width=1.5cm]{img/cka-certified-kubernetes-administrator.png}}
        
        \ecvblueitem{Febbraio 2021}{AWS Certified Solution Architect - Associate (SAA-CO2)}
        \ecvitem[.2cm]{Ente certificatore}{Amazon Web Services (AWS)}
        \ecvitem{Identificativo}{2898P6CBJ111QMWC (http://aws.amazon.com/verification)}\hfill
        \smash{\includegraphics[width=1.5cm]{img/aws-certified-solutions-architect-associate.png}}       
   

  \ecvsection{Educazione e formazione}
  
  \ecvtitlelevel{2017 -- 2020}{Laurea magistrale in scienze e tecnologie informatiche}{ISCED~7}
  \ecvitem{}{Università degli studi di Torino}
  \ecvitem[.2cm]{Votazione}{ 110/110 con Lode e menzione }
  \ecvitem{Tesi}{Identity and Access governance : gestione delle identità digitali all'interno di un'azienda.}
  
  \ecvtitlelevel{2014 -- 2017}{Laurea triennale in scienze e tecnologie informatiche}{ISCED~6}
  \ecvitem{}{Università degli studi di Torino}
  \ecvitem[.2cm]{Votazione}{ 110/110 con Lode}
  \ecvitem{Tesi}{Vulnerability assessment di applicazioni Android: uno strumento per l'automazione.}
  
  
  \ecvtitlelevel{2009 -- 2014}{Diploma di perito informatico}{ISCED~4}
  \ecvitem{}{ITIS Giuseppe Peano, Torino}
  \ecvitem[.2cm]{Votazione}{ 75/100}    
  
  



  \ecvsection{Competenze personali}
  \ecvmothertongue{Italiano}
  \ecvlanguageheader
  \ecvlanguage{Inglese}{B2.2}{B2.2}{B2.2}{B2.2}{B2.2}
  \ecvlanguagecertificate{Speexx Language Assessment (05/2021)}
  \ecvlanguagecertificate{Preliminary English Test (07/2014)}

 % \ecvlanguagefooter
   
  \ecvblueitem{Comunicazione}{
  Ottima capacità di lavorare in team iniziata sin da bambino praticando per diversi anni sport di squadra, otre che nei diversi progetti svolti sia durante il percorso universitario che in ambito lavorativo in ogni lavoro svolto.
  }
  
  
  
  \ecvblueitem{Competenze digitali}{
    \begin{ecvitemize}
        \item Buona conoscenza del cloud AWS e dei relativi servizi, con particolare esperienza su servizi inerenti a tematiche di automazione e DevOps (\textit{CloudFormation}, \textit{CodePipeline}, \textit{CodeDeploy...}); 
        \item Ottima conoscenza di \textit{Kubernetes} e di tool correlati come \textit{Helm} e \textit{Kustomize}.
        \item Buona conoscenza di strumenti per \textit{Infrastructure as a Code}(IAC) tra cui \textit{Terraform}, \textit{CloudFormation} (e \textit{SAM}) ed il framework \textit{Serverless} ;
        \item Buona conoscenza del paradigma serverless, del paradigma a microservizi e sui container (in particolare su \textit{Docker}), sulla loro orchestrazione (\textit{Kubernetes}, \textit{EKS} , \textit{ECS} ) e relativo rilascio e monitoraggio su ambienti produttivi (\textit{Prometheus} , \textit{Grafana}) ;
        \item Buona conoscenza di \textit{Git}, del paradigma \textit{GitOps} ed esperienza nella creazione di processi di CI/CD tramite strumenti di automazione tra cui in particolare \textit{Jenkins} (sfruttando l'approccio scalabile \textit{Jenkins on Kubernetes}), oltre che Atlassian \textit{Bamboo} e \textit{GitLab} .
        \item Ottime competenze in ambito sviluppo software ed in particolare nel linguaggio \textit{Java}. Buona conoscenza di diversi altri linguaggi tra cui \textit{Kotlin} (+ SDK Android), C (+ OpenMP/MPI) e JS (+ React). Buone competenze in ambito scripting tramite \textit{Bash} o \textit{Python}.
        \item Buona conoscenza dei RDBMS (e del linguaggio SQL)
        \item Conoscenza base di \textit{OpenStack}
        \item Eseprienza nell'utilizzo di sistemi Linux
        \item Buona conoscenza in ambito networking (sia in ambito cloud che non) e dello stack TCP/IP.
        
    \end{ecvitemize}
  }

  
  
    \ecvsection{Ulteriori informazioni}
    
    
        \ecvblueitem{Patente}{Patente B}
        
        \ecvblueitem{Hobbies e interessi}{
            Relativamente al lavoro, mi interesso verso le nuove tecnologie ed in generale tutto ciò che c'è di nuovo. Nel tempo libero amo fare sport, in particolare sciare, trekking o giocare a calcio. Inoltre mi piace molto leggere, giocare a scacchi e fare giardinaggio.
        }
        
        \ecvblueitem{Trattameto dei dati}{Autorizzo il trattamento dei miei dati personali ai sensi del Decreto Legislativo 30 giugno 2003, n. 196 "Codice in materia di protezione dei dati personali.}

  
  \end{europasscv}

\end{document}

