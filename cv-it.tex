% !TEX encoding = UTF-8
% !TEX program = pdflatex
% !TEX spellcheck = it_IT

\documentclass[italian,a4paper]{europasscv}
\usepackage[italian]{babel}
\usepackage{comment}
\usepackage{lastpage}
\usepackage{hyperref}
\usepackage{fontawesome5}

\ecvname{Olla Gabriele}


\begin{document}

\begin{europasscv}


\ecvtitle{Nome e Cognome}{{{\LARGE Olla Gabriele}}}
\vspace{0.2cm}
\ecvitem{Data di Nascita}{17/02/1995} 
\ecvitem{Nazionalità}{Italiana}
\ecvitem{Mail}{\href{mailto:jobs@ollaw.xyz}{jobs@ollaw.xyz} \href{mailto:ollagabriele@proton.me}{ollagabriele@proton.me}}
\ecvitem{Link}{ \textcolor{ecvrulecolor}{\faLinkedin} \href{https://linkedin.com/in/ollagabriele}{ollagabriele} \quad \textcolor{ecvrulecolor}{\faGitlab} \href{https://gitlab.com/ollaww}{ollaww} \quad \textcolor{ecvrulecolor}{\faGithub} \href{https://github.com/ollaw}{ollaw}}

	
\ecvsection{Esperienza Lavorativa}

\ecvtitle{05/2022 -- Presente}{Site Reliability Engineer}
\ecvitem{Azienda}{\textbf{Prima Assicurazioni}, Milano [\textit{Da remoto}]}
\ecvitem{Attività principali}{%
  \begin{ecvitemize}
    \item Progettazione e gestione dell’infrastruttura AWS tramite Pulumi e Terraform.
    \item Introduzione di HashiCorp Vault come secret manager centralizzato.
    \item Integrazione di Okta con gli strumenti aziendali (GitHub, Vault, Datadog, RabbitMQ).
    \item Manutenzione dell’infrastruttura CI/CD con GitHub Actions e supporto ai team di sviluppo.
    \item Migrazioni di cluster RabbitMQ e ottimizzazione dei costi cloud.
  \end{ecvitemize}
}
\vspace{0.2cm}

\ecvtitle{09/2021 -- 05/2022}{Cloud Engineer}
\ecvitem{Azienda}{\textbf{DXC Technology}, Cernusco sul Naviglio (MI) [\textit{Ibrido}]}
\ecvitem{Attività principali}{%
  \begin{ecvitemize}
    \item Provisioning di cluster Kubernetes basati su MKE e RKE.
    \item Creazione di infrastrutture su OpenStack utilizzando Terraform.
    \item Configurazione automatizzata di macchine virtuali tramite Ansible.
  \end{ecvitemize}
}
\vspace{0.2cm}

\ecvtitle{03/2020 -- 08/2021}{Cloud Engineer}
\ecvitem{Azienda}{\textbf{Storm Reply}, Torino [\textit{Ibrido}]}
\ecvitem{Attività principali}{%
  \begin{ecvitemize}
    \item Gestione e scaling di Jenkins su Kubernetes per pipeline CI/CD.
    \item Migrazioni su AWS con provisioning infrastrutturale via CloudFormation e Terraform.
    \item Gestione dei processi di rilascio su EKS ed EC2 (Linux/Windows) tramite CodeDeploy.
    \item Setup e configurazione di uno stack ELK per il monitoraggio centralizzato.
  \end{ecvitemize}
}
\vspace{0.2cm}

\ecvsection{Certificazioni}


    \ecvitem{Linux Foundation, 2025}{FinOps Certified Engineer} 
    \ecvitem{Hashicorp, 2025}{HashiCorp Certified: Vault Associate (003)} 
    \ecvitem{Linux Foundation, 2023}{Certified Kubenretes Security Specialist (CKS)} 
	\ecvitem{Linux Foundation, 2021}{Certified Kubernetes Application Developer (CKAD)}
	\ecvitem{Linux Foundation, 2021}{Certified Kubernetes Administrator (CKA)}        
	\ecvitem{AWS, 2021}{AWS Solution Architect (SAA-CO2)}



	           
\ecvsection{Istruzione}
	  
\ecvtitlelevel{2017 -- 2020}{Laurea Magistrale in Informatica e Tecnologie}{ISCED~7}
\ecvitem{}{\textit{Università degli studi di Torino}}
\ecvitem[.2cm]{Votazione}{ 110/110 con lode}
\ecvitem{Tesi}{Identity and Access governance: gestione delle identità digitali all'interno di un'azienda.}
	  
	  
\ecvtitlelevel{2014 -- 2017}{Laurea Triennale in Informatica e Tecnologie}{ISCED~6}
\ecvitem{}{\textit{Università degli studi di Torino}}
\ecvitem[.2cm]{Votazione}{ 110/110 con lode}
\ecvitem{Tesi}{Vulnerability assessment di applicazioni Android: uno strumento per l'automazione.}
	
 
\ecvsection{Competenze Tecniche}


\ecvblueitem{Cloud Computing}{Solida esperienza nella progettazione e gestione di infrastrutture su Amazon Web Services (AWS), con particolare attenzione alla scalabilità, sicurezza e ottimizzazione dei costi. Responsabile della gestione di ambienti multi-account con strutture organizzative centralizzate e controllo granulare degli accessi. Competenza avanzata nell'uso di strumenti di Infrastructure as Code (IaC) come Pulumi (per la gestione moderna delle risorse cloud) e Terraform.}

\ecvblueitem{IAM \& Secrets Management}{Responsabile dell’introduzione e gestione di soluzioni IAM complesse in ambienti enterprise. Ampia esperienza nell’integrazione di Okta come Identity Provider (IdP) e Single Sign-On (SSO) per strumenti critici (AWS, GitHub, Vault, RabbitMQ, Artifactory [...]). Progettazione e adozione di HashiCorp Vault come secret manager centralizzato, includendo la definizione di policy, l’integrazione con Kubernetes e la gestione sicura del ciclo di vita dei segreti.}

\ecvblueitem{Container e orchestrazione}{Competenza avanzata con container e orchestratori. Gestione e provisioning di cluster Kubernetes sia gestiti (EKS) che self-hosted (MKE, RKE), con esperienza in scenari multi-cluster e automazione del deployment. Esecuzione di migrazioni di carichi di lavoro e integrazione con strumenti esterni come Vault e Okta.}

\ecvblueitem{Messaging \& Event Systems}{Esperienza diretta nella gestione e migrazione di cluster RabbitMQ, inclusa la federazione tra ambienti e la gestione delle configurazioni su CloudAMQP in ambienti multi-tenant. Integrazione di RabbitMQ con sistemi di autenticazione centralizzati e monitoring.}

\ecvblueitem{Tooling \& DevOps}{Gestione dell’infrastruttura CI/CD con GitHub Actions. Setup e mantenimento di strumenti per feature flagging (LaunchDarkly) e observability (Datadog). Esperienza nell’automazione di pipeline, nei processi di rilascio su ambienti containerizzati e nell’adozione di metriche per il monitoraggio e l’ottimizzazione delle performance.}

\ecvblueitem{Competenze Aggiuntive}{Buona padronanza di Python, Java e Kotlin. Conoscenze di base in Go e C (anche in ambito parallelo con OpenMP/MPI) e JavaScript con competenze in frontend (React/Vue). Esperienza nello sviluppo mobile in ambito accademico con Android SDK. Ottime capacità di collaborazione in team distribuiti e autonomia nel lavoro remoto.}

	
\ecvsection{Lingue}
   \ecvmothertongue{Italiano}
   \ecvblueitem{Altre Lingue}{Inglese (C1 - Fluentify Language Assessment 12/2022)}
	

   \ecvsection{Informazioni Aggiuntive}
	    
   \ecvblueitem{Patente di Guida}{A, B}
	        
   \ecvblueitem{Trattamento dei Dati}{Autorizzo il trattamento dei miei dati personali in conformità al Decreto Legislativo 30 giugno 2003, n. 196 "Codice in materia di protezione dei dati personali".}
	

		  
\end{europasscv}

\end{document}

