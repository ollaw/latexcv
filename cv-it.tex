% !TEX encoding = UTF-8
% !TEX program = pdflatex
% !TEX spellcheck = it_IT

\documentclass[italian,a4paper]{europasscv}
\usepackage[italian]{babel}
\usepackage{comment}
\usepackage{lastpage}
\usepackage{hyperref}
\ecvname{Olla Gabriele}
\ecvaddress{Torino (TO)}

\ecvemail{jobs@ollaw.xyz}
\ecvlinkedinpage{https://www.linkedin.com/in/ollagabriele}
\ecvgitlabpage{ollaww}
\ecvgithubpage{ollaw}

\ecvdateofbirth{17/02/1995}
\ecvnationality{Italiana}


\begin{document}
\begin{europasscv}
		
\ecvpersonalinfo
	
\ecvsection{Esperienza Lavorativa}
	  
\ecvtitle{Dal 05/2022}{Site Reliability Engineer (precedentemente DevOps Engineer)}
\ecvitem{Azienda}{\textbf{Prima Assicurazioni}, Milano [\textit{Completamente remoto}]}
\ecvitem{Tecnologie}{AWS, Kubernetes, Helm, Pulumi, GitHub Actions, Okta, Datadog, Vault.}
\vspace{0.5cm}
	  
\ecvtitle{09/2021 -- 05/2022}{Cloud Engineer}
\ecvitem{Azienda}{\textbf{DXC Technology}, Cernusco sul Naviglio (MI) [\textit{Lavoro ibrido}]}
\ecvitem{Tecnologie}{Kubernetes (MKE, RKE), Helm, Docker Swarm, Gitlab, Terraform, Ansible.}

\vspace{0.5cm}
	
\ecvtitle{3/2020 -- 08/2021}{Cloud Engineer}
\ecvitem{Azienda}{\textbf{Storm Reply}, Torino [\textit{Lavoro ibrido}]}
\ecvitem{Tecnologie}{AWS, Kubernetes, Jenkins, ELK.}
\vspace{0.5cm}

	           
\ecvsection{Istruzione}
	  
\ecvtitlelevel{2017 -- 2020}{Laurea Magistrale in Informatica e Tecnologie}{ISCED~7}
\ecvitem{}{\textbf{Università di Torino}}
\ecvitem[.2cm]{Voto}{ 110/110 con lode}
\ecvitem{Tesi}{Identity and Access governance: gestione delle identità digitali all'interno di un'azienda.}
	  
	  
\ecvtitlelevel{2014 -- 2017}{Laurea Triennale in Informatica e Tecnologie}{ISCED~6}
\ecvitem{}{\textbf{Università di Torino}}
\ecvitem[.2cm]{Voto}{ 110/110 con lode}
\ecvitem{Tesi}{Vulnerability assessment di applicazioni Android: uno strumento per l'automazione.}
	
 
\ecvsection{Competenze Tecniche}

    \ecvblueitem{Cloud Computing}{Vasta esperienza nel cloud, in particolare su Amazon Web Services (AWS). Competenze approfondite nella progettazione, implementazione e migrazione di infrastrutture, specializzazione nella configurazione e gestione di AWS Organizations tramite Identity Center per una governance efficiente delle risorse cloud. Conoscenza avanzata di strumenti di Infrastructure as Code (IaC) come Pulumi e Terraform.}

    \ecvblueitem{IAM \& Secrets Management}{Ampia esperienza nell'implementazione e configurazione di soluzioni IAM su AWS, inclusa l'ampia utilizzazione di Okta come Identity Provider (IdP) e la configurazione di Okta come soluzione centralizzata di Single Sign-On (SSO) per una vasta gamma di applicazioni critiche come RabbitMQ (CloudAMQP), Artifactory, AWS e Vault. Inoltre, esperienza nella configurazione ed integrazione di Vault per la gestione sicura dei segreti e dei cicli di vita delle chiavi di crittografia, garantendo la confidenzialità e l'integrità dei dati sensibili nell'ambiente cloud.}

    \ecvblueitem{Container e orchestrazione}{Eccellente competenza nell'uso dei container e conoscenza approfondita di Kubernetes, sia in ambienti gestiti (come EKS) che auto-gestiti (MKE e RKE). Esperienza consolidata nella migrazione di carichi di lavoro di produzione tra cluster diversi e nell'integrazione con strumenti di terze parti, come Okta e Vault.}

    \ecvblueitem{Competenze Aggiuntive}{Padronanza di Python, Java e Kotlin, con esperienza pratica nell'implementazione di progetti software. Conoscenza sufficiente di linguaggi come Go e C (con esperienza in OpenMP/MPI) e JavaScript (con competenze in React/Vue). Esperienza universitaria nello sviluppo mobile utilizzando Android SDK. Abilità dimostrate nel lavorare in team, collaborando efficacemente con colleghi per raggiungere obiettivi comuni. Abilità dimostrate nel lavoro remoto, garantendo produttività e qualità del lavoro anche in contesti distribuiti.}
	


\ecvsection{Certificazioni}

    \ecvblueitem{Novembre 2023}{Certified Kubenretes Security Specialist (CKS)}
	\ecvitem[.2cm]{Ente Certificatore}{Cloud Native Computing Foundation (CNCF)}
	\ecvitem{ID}{ LF-mf4uknkmry (https://training.linuxfoundation.org/certification/verify/)}\hfill
	\smash{\includegraphics[width=1.5cm]{img/cks.png}}
	    
	\ecvblueitem{Dicembre 2021}{Certified Kubernetes Application Developer (CKAD)}
	\ecvitem[.2cm]{Ente Certificatore}{Cloud Native Computing Foundation (CNCF)}
	\ecvitem{ID}{ LF-owlpgjxmun (https://training.linuxfoundation.org/certification/verify/)}\hfill
	\smash{\includegraphics[width=1.5cm]{img/ckad.png}}
		        
	\ecvblueitem{Novembre 2021}{Certified Kubernetes Administrator (CKA)}
	\ecvitem[.2cm]{Ente Certificatore}{Cloud Native Computing Foundation (CNCF)}
	\ecvitem{ID}{ LF-4w8nbsr5xc (https://training.linuxfoundation.org/certification/verify/)}\hfill
	\smash{\includegraphics[width=1.5cm]{img/cka.png}}
		        
	\ecvblueitem{Febbraio 2021}{AWS Solution Architect (SAA-CO2)}
	\ecvitem[.2cm]{Ente Certificatore}{Amazon Web Services (AWS)}
	\ecvitem{ID}{2898P6CBJ111QMWC (http://aws.amazon.com/verification)}\hfill
	\smash{\includegraphics[width=1.5cm]{img/aws-certified-solutions-architect-associate.png}}

\ecvsection{Lingue}
   \ecvmothertongue{Italiano}
   \ecvblueitem{Altre Lingue}{Inglese (C1 - Fluentify Language Assessment 12/2022)}
	

   \ecvsection{Informazioni Aggiuntive}
	    
   \ecvblueitem{Patente di Guida}{Categoria B}
	        
   \ecvblueitem{Interessi e Hobby}{
        Nel tempo libero, mi piacciono vari sport, in particolare sciare, fare trekking o giocare a calcio. Mi piace anche leggere, giocare a scacchi, mantenere un acquario e fare giardinaggio.
    }
	        
   \ecvblueitem{Trattamento dei Dati}{Autorizzo il trattamento dei miei dati personali in conformità al Decreto Legislativo 30 giugno 2003, n. 196 "Codice in materia di protezione dei dati personali".}
	
	  

		  
\end{europasscv}

\end{document}

