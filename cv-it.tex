% !TEX encoding = UTF-8
% !TEX program = pdflatex
% !TEX spellcheck = en_GB

\documentclass[italian,a4paper]{europasscv}
\usepackage[italian]{babel}

\ecvname{Olla Gabriele}
\ecvaddress{Via Nizza 356bis, Torino (TO)}
\ecvmobile{+39 348 6011 931}

\ecvemail{ollagabriele@pm.me}
\ecvgitlabpage{https://gitlab.com/ollaww}
\ecvlinkedinpage{https://www.linkedin.com/in/ollagabriele}

\ecvgender{Uomo}
\ecvdateofbirth{17/02/1995}
\ecvnationality{Italiana}

% \ecvpicture[width=3.8cm]{picture.jpg}


\begin{document}
  \begin{europasscv}

  \ecvpersonalinfo


  %\ecvbigitem{Job applied for}{Insert job}
  
  %\ecvsection{Personal statement}
  %    \ecvblueitem{}{TODO}

  \ecvsection{Esperienze lavorative}
  
  \ecvtitle{Settembre 2021 -- Attuale}{Cloud Engineer}
  \ecvitem{}{DXC Technology, Cernusco sul Naviglio (MI)}
  \ecvitem[2.5mm]{}{
  \begin{ecvitemize}
        \item Creazione di infrastrutture cloud su cloud provider pubblici (AWS, Azure, GCP) e privati (Openstack) ;
        \item Gestione di processi CI/CD tramite tool aziendali basati su Docker Swarm e Kubernetes (MKE)
      \end{ecvitemize}
  }

  
  \ecvtitle{Marzo 2020 -- Agosto 2021}{AWS Cloud Engineer}
  \ecvitem{}{Storm Reply, Torino}
    \ecvitem[2.5mm]{}{
      \begin{ecvitemize}
        \item Migrazione e creazione di infrastrutture cloud mediante l'utilizzo di tool per IAC (CloudFormation, SAM, Serverless Framework, Terraform);
        \item Design e implementazione di processi DevOps ( CI/CD, Monitoring ...)  per i rilasci in produzione di diverse tipologie di applicativi su cloud, sfruttando particolarmente Jenkins e Kubernetes (EKS).
        \item Sistemi di monitoring e logging sulle inrfastrutture cloud.
      \end{ecvitemize}
      }
   
  \ecvtitle{Settembre 2019 -- Marzo 2020}{Security Consultant (Internship)}
  \ecvitem{}{Accenture Security, Milano}
  \ecvitem[2.5mm]{}{
  \begin{ecvitemize}
      \item Analisi per l'implementazione di sistemi di Identity and Access Governance (IAG) ;
      \item Integrazione e competenza pratica nell'utilizzo della piattaforma \textit{Sailpoint Identity IQ};
      \item Sviluppo di \textit{plugin} e funzionalità (Java/Beanshell) al di sopra della piattaforma per facilitare l'integrazione con le applicazioni esistenti nell'azienda.
  \end{ecvitemize}
  }

  
  \ecvsection{Educazione e formazione}
  
  \ecvtitlelevel{Settembre 2017-- Marzo 2020}{Laurea magistrale in scienze e tecnologie informatiche}{ISCED~7}
  \ecvitem{}{Università degli studi di Torino}
  \ecvitem[.2cm]{Votazione}{ 110/110 con Lode e menzione }
  \ecvitem{Tesi}{Identity and Access governance : gestione delle identità digitali all'interno di un'azienda.}
  
  \ecvtitlelevel{Settembre 2014-- Ottobre 2017}{Laurea triennale in scienze e tecnologie informatiche}{ISCED~6}
  \ecvitem{}{Università degli studi di Torino}
  \ecvitem[.2cm]{Votazione}{ 110/110 con Lode}
  \ecvitem{Tesi}{Vulnerability assessment di applicazioni Android: uno strumento per l'automazione.}
  
  
  \ecvtitlelevel{2009-2014}{Diploma di perito informatico}{ISCED~4}
  \ecvitem{}{ITIS Giuseppe Peano, Torino}
  \ecvitem[.2cm]{Votazione}{ 75/100}    
  
  \ecvsection{Certificazioni}
  \ecvblueitem{Dicembre 2021}{Certified Kubernetes Application Developer (CKAD)}
  \ecvitem[.2cm]{Ente certificatore}{Cloud Native Computing Foundation (CNCF)}
  \ecvitem{Identificativo}{ LF-owlpgjxmun (https://training.linuxfoundation.org/certification/verify/)}\hfill
  \smash{\includegraphics[width=1.5cm]{img/ckad-certified-kubernetes-application-developer.png}}

  \ecvblueitem{Novembre 2021}{Certified Kubernetes Administrator (CKA)}
  \ecvitem[.2cm]{Ente certificatore}{Cloud Native Computing Foundation (CNCF)}
  \ecvitem{Identificativo}{ LF-4w8nbsr5xc (https://training.linuxfoundation.org/certification/verify/)}\hfill
  \smash{\includegraphics[width=1.5cm]{img/cka-certified-kubernetes-administrator.png}}

  \ecvblueitem{Febbraio 2021}{AWS Certified Solution Architect - Associate (SAA-CO2)}
  \ecvitem[.2cm]{Ente certificatore}{Amazon Web Services (AWS)}
  \ecvitem{Identificativo}{2898P6CBJ111QMWC (http://aws.amazon.com/verification)}\hfill
  \smash{\includegraphics[width=1.5cm]{img/aws-certified-solutions-architect-associate.png}}



  \ecvsection{Competenze personali}
  \ecvmothertongue{Italiano}
  \ecvlanguageheader
  \ecvlanguage{Inglese}{B2.2}{B2.2}{B2.2}{B2.2}{B2.2}
  \ecvlanguagecertificate{Speexx Language Assessment (05/2021)}
  \ecvlanguagecertificate{Preliminary English Test (07/2014)}

 % \ecvlanguagefooter
   
  \ecvblueitem{Comunicazione}{
  Ottima capacità di lavorare in team iniziata sin da bambino praticando per diversi anni sport di squadra, otre che nei diversi progetti svolti sia durante il percorso universitario che in ambito lavorativo in ogni lavoro svolto.
  }
  
  
  
  \ecvblueitem{Competenze digitali}{
    \begin{ecvitemize}
        \item Forte conoscenza del cloud AWS e dei relativi servizi, con particolare esperienza su servizi inerenti a tematiche di automazione e DevOps (\textit{CloudFormation}, \textit{CodePipeline}, \textit{CodeDeploy...}); 
        \item Ottima conoscenza di \textit{Kubernetes} e di tool correlati come \textit{Helm} e \textit{Kustomize}.
        \item Buona conoscenza di strumenti per \textit{Infrastructure as a Code}(IAC) tra cui \textit{Terraform}, \textit{CloudFormation} (e \textit{SAM}) ed il framework \textit{Serverless} ;
        \item Buona conoscenza del paradigma serverless, del paradigma a microservizi e sui container (in particolare su \textit{Docker}), sulla loro orchestrazione (\textit{Kubernetes}, \textit{EKS} , \textit{ECS} ) e relativo rilascio e monitoraggio su ambienti produttivi (\textit{Prometheus} , \textit{Grafana}) ;
        \item Buona conoscenza di \textit{Git}, del paradigma \textit{GitOps} ed esperienza nella creazione di processi di CI/CD tramite strumenti di automazione tra cui in particolare \textit{Jenkins} (sfruttando l'approccio scalabile \textit{Jenkins on Kubernetes}), oltre che Atlassian \textit{Bamboo} e \textit{GitLab} .
        \item Ottime competenze in ambito sviluppo software ed in particolare nel linguaggio \textit{Java}. Buona conoscenza di diversi altri linguaggi tra cui \textit{Kotlin} (+ SDK Android), C (+ OpenMP/MPI) e JS (+ React). Buone competenze in ambito scripting tramite \textit{Bash} o \textit{Python}.
        \item Buona conoscenza dei RDBMS (e del linguaggio SQL)
        \item Conoscenza base di \textit{OpenStack}
        \item Eseprienza nell'utilizzo di sistemi Linux
        \item Buona conoscenza in ambito networking (sia in ambito cloud che non) e dello stack TCP/IP.
        
    \end{ecvitemize}
  }

  \ecvblueitem{Patente}{Patente B}
  
\ecvsection{Ulteriori informazioni}
\ecvblueitem{Trattameto dei dati}{Autorizzo il trattamento dei miei dati personali ai sensi del Decreto Legislativo 30 giugno 2003, n. 196 "Codice in materia di protezione dei dati personali.}

  
  \end{europasscv}

\end{document}
